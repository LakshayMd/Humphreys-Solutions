\chapter{Solvable and nilpotent Lie algebras}

\begin{solution}{}{3.1}
    
\end{solution}

\begin{solution}{}{3.2}

\end{solution}

\begin{solution}{}{3.3}
    The relations
    \begin{align*}
        [x,y] &= h \\
        [h,x] &= 2x \\
        [h,y] &= -2y
    \end{align*}
    hold equally well in characteristic 2, but the last 2 brackets are then 0. Thus $ L^{1} = [L, L] = \mathbb{C}h $, and $ L^{2} = [L^{1}, L] = [ \mathbb{C}h, L] = 0 $, since $ [h, -] $ is uniformly 0.
\end{solution}

\begin{solution}{}{3.4}

\end{solution}

\begin{solution}{}{3.5}
    
\end{solution}

\begin{solution}{}{3.6}

\end{solution}

\begin{solution}{}{3.7}
    $ K \subset N_{L}(K) $ follows from $ K $ being closed under the bracket operation. Let $ n $ be the unique non-negative integer such that $ L^{n} \subsetneq K $ but $ L^{n+1} \subset K $, which exists because $ K $ is proper and $ L $ is nilpotent. Let $ z\in L^{n} \setminus K $. I claim that $ z\in N_{L}(K) $. Indeed,
    \begin{align*}
        zK \subset [L^{n}, K]
        \subset [L^{n}, L] = L^{n+1}
        \subset K
    \end{align*}
\end{solution}

\begin{solution}{}{3.8}

\end{solution}

\begin{solution}{}{3.9}
    
\end{solution}

\begin{solution}{}{3.10}

\end{solution}